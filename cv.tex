\documentclass{my_cv}
\begin{document}

\name{Stephen Polyak, Ph.D.}
\\
\vspace{0.5cm}
\\
\contact{Iowa City, IA}{USA}{steve.polyak@clinicalink.com}{linkedin.com/in/stevepolyak}{(+1)-319-621-2749}
%\longcontact{123 Broadway}{London}{UK 12345}{john@smith.com}{(000)-111-1111}


\section{Objective}
\hspace{1pt}\parbox{0.99\textwidth}{
An experienced architecture leader looking for opportunities to design, communicate and deliver software solutions. 
}

%\vspace{-7pt}
\\
\vspace{0.3cm}
\\
\section{Education \hfill  {\small \href{https://www.researchgate.net/profile/Stephen-Polyak}{\includegraphics[scale=0.4]{edinburgh.jpeg}}}}

%\href{https://www.dai.ed.ac.uk/papers/authors/stevep.html}

\edsubsection{University of Edinburgh}{Edinburgh, Scotland UK}{Ph.D. in Artificial Intelligence}{1996--2000}{AI Planning: An Integration Framework for Managing Rich Organisational Process Knowledge}\vspace{0.1cm}
\\
\edsubsection{DePaul University}{Chicago, IL USA}{M.Sc. in Computer Science}{1992--1994}
\\
\vspace{0.1cm}
\\
\edsubsection{University of Iowa}{Iowa City, IA USA}{B.Sc. in Psychology}{1987--1991}

\vspace{-15pt}

\\
\vspace{0.5cm}
\\

\section{Work Experience \hfill  {\small \href{https://linkedin.com/in/stevepolyak}{\includegraphics[scale=0.075]{LI-Logo.png}}}}
\worksubsection{Clinical ink}{Iowa City, IA USA}{Vice President, Engineering and Data}{November 2021--Present}{
    \item[\textbf{--}] Currently leading an advanced, multidisciplinary technology team (25 members) of software developers, data engineers, scientists, project managers, and product/UX designers. 
    \item[\textbf{--}] Delivered advanced sensor and wearable solutions for several international clinical trials using provisioned and BYOD consumer grade products (iPhone, iPad, Apple Watch, Android phone).
    \item[\textbf{--}] Solution development for numerous connected device studies, including observation, phase 2, phase 3, and extension studies for therapeutic areas including Parkinson's Disease, ALS, COVID-19/Flu, Hemophilia, ADHD, Multiple System Atrophy (MSA).
    \item[\textbf{--}] Research \& Development, clinical trial innovation proof of concepts for diabetes/weight loss trial data collection (CGM/BGM), telehealth, predicting study adherence (Machine learning), environmental data (pollen, air quality, weather), augmented reality body measurements.  
    \item[\textbf{--}] Amazon Web Services, iOS/Android, React Native, React, Python, Snowflake, H2O AutoML, AWS SageMaker.
    % \item[\textbf{--}] 
}\vspace{0.1cm}
\\
\worksubsection{Digital Artefacts}{Iowa City, IA USA}{Vice President, Engineering and Data}{August 2020 - October 2023}{
    \item[\textbf{--}] Lead engineering and data team at Digital Artefacts offering custom software, mobile, web, and hardware development consulting. Delivering transformational digital products for the Fortune 500, the federal government, award-winning start-ups, and research institutions. 
    \item[\textbf{--}] Personnel interview/hiring, customer solution communication/management.
    \item[\textbf{--}] Advanced international clinical trials product development based on the BrainBaseline platform.
    \item[\textbf{--}] Digital Artefacts was acquired by Clinical ink in a November 2021 Merger \& Acquisition and continued to operate as its bespoke, wearables and sensors solution department to present.   
    % \item[\textbf{--}] 
}\vspace{0.1cm}
\\
%\pagebreak
}
\worksubsection{ACT, Inc.}{Iowa City, IA USA}{Senior Director, Research Innovation Development}{August 2017 -- August 2020}{
 \item[\textbf{--}] Area leadership: people (15 FTE/contractors), strategy and budget management, capacity planning, personnel interview/hiring, feature solution designs.
 \item[\textbf{--}] Lead agile teams of software developers, testers, data engineers, architects, mid-level managers as part of ACT’s research and development unit, ACTNext.
 \item[\textbf{--}] Established a “raw data to insights” capability using Databricks (PySpark/R/SQL) to manage Kinesis streams of data, address data privacy/access and enable elastic spark cluster ETL pipelines replacing ad hoc, manual research and test security processes.
 \item[\textbf{--}] Created ACT’s first voice-based product, ACT Alexa Skill, in partnership with Amazon Education using Alexa SDK/developer console, Node.js, AWS Lambda, API Gateway, DynamoDB integrated with ACT enterprise APIs that enables access to test center data and test preparation with ACT Academy. 
 \item[\textbf{--}] Lead/Built ACTNext’s Recommendations and Diagnostic (RAD) API using Swagger, Node.js, AWS Lambda, API Gateway, DynamoDB with advanced psychometrics, learning solutions, AI/ML teams. 
 \item[\textbf{--}]Integrated the API with ACT Academy and a pilot partner Alchemie to accept learning events and manage continuous estimations of mastery.
 \item[\textbf{--}] Lead/Built RAD Dashboard webapp using Angular, Chartist.js, AWS Cognito, AWS Lambda to provide a multi-tenant view into the RAD API data.

}\vspace{0.1cm}
\\
\worksubsection{}{Iowa City, IA USA}{Principal Architect Innovation and Assessment Design}{September 2014 - August 2017}{
    \item[\textbf{--}] Lead/Built the ACTNext companion mobile application using React Native, Ignite CLI, fastlane, Amazon Mobile Analytics (now Pinpoint), AWS Lambda, Node.js, Apple TestFlight as a front-end to the RAD API providing targeted learning resources to users. Conducted pilot study with partner high school on Apple iPads (N=100).
    \item[\textbf{--}] Established ACTLabs at ACT. ACTLabs provides virtual (AWS-based services) and physical (interactive lab space, access to VR, mobile devices, cameras, eyetracking equipment, sensors) to support innovative prototype development.
    \item[\textbf{--}] Office for Innovation leadership developing several web prototypes (D3.js, Vue.js) exploring a variety of concepts taken from our ideation platform.
}
\vspace{0.1cm}
\\
\worksubsection{}{Iowa City, IA USA}{Lead Architect}{April 2013 – September 2014}{
    \item[\textbf{--}] Member of the technical architecture team responsible for strategic direction of ACT’s application portfolio. Developed several solution designs for ACT products and services as well as integrations with external partners.
}
\vspace{0.1cm}
\\
\worksubsection{Pearson Educational Measurement}{Iowa City, IA USA}{Principal Architect}{August 2002 - April 2013}{
    \item[\textbf{--}] Worked on/developed many critical projects at Pearson, e.g. ePEN, a plan for a Unified System Strategy (USS) through the conceptualization/deployment of PearsonAccess and the establishment of an enterprise architecture where I was part of a core team that worked on solidifying the architecture and implementation for the next generation of the web platform PearsonAccess (Next) (Java, Spring).
}
\vspace{0.1cm}
\\
\worksubsection{Encyclopaedia Britannica}{Chicago, IL USA}{Director of Site Technologies}{1999 - 2002}{
    \item[\textbf{--}] Led conversion of the www.britannica.com site from Vignette to Weblogic (Java2EE, XML/XSLT, Oracle) for deployment at ASP, Responsible for a team of 10 developers, Created several core software components (XSL stylesheets, Taglibs).
}
\vspace{0.1cm}
\\
\worksubsection{Artificial Intelligence Applications Institute}{Edinburgh, Scotland UK}{AI Ph.D. Student}{1996 - 1999}{
    \item[\textbf{--}] 18 papers co/authored, accepted for journals, book, conference, workshops or as departmental technical reports. Developed the common process ontology, framework, and language.
    \item[\textbf{--}] Paper referee for international journals and workshops: IEEE Expert Intelligent Systems, Knowledge and Information Systems (KAIS), SMBPI, Encyclopedia of Cognitive Science (MIT Press).
    \item[\textbf{--}] Collaborated with other University/Governmental/Corporate researchers on International Standard for a Process Interchange Format (PIF), DARPA/ARPI Shared Planning and Activity Representation (SPAR). Research member with the Software Systems and Processes Group (SSP).
    \item[\textbf{--}] Primary Investigator for the United States Department of Commerce project, “Scenario Development for Shared Process Models” and joint PI on the United States National Institute of Standards and Technology (NIST) Project for the Process Specification Language Programme.
}
\vspace{0.1cm}
\\
\worksubsection{Accenture/Computer Sciences Corp (CSC)/Access Health}{Chicago, IL USA}{Consultant/Engineer/Programmer}{1993 - 1996}{
    \item[\textbf{--}]  Various software development positions in the Chicagoland area working on Point of Sale systems, expert systems, call center software.
}

\vspace{-12pt}

\section{Publications (Selected)  \hfill { \small \href{https://scholar.google.com/citations?user=iWrLGjMAAAAJ&hl=en}{\includegraphics[scale=0.15]{scholar_logo_64dp.png}}}}
% \parbox{\textwidth}{
\begin{itemize}[leftmargin=10pt]
    % \setlength{\itemsep}{0pt}

    \item[\textbf{--}] Dr. David Anderson, Joan Severson , Dr E. Dorsey , Jamie Adams , Ms Tairmae Kangarloo , Ms Melissa Kostrzebski , Mr Allen Best , Michael Merickel , Dan Amato , Brian Severson , Sean Jezewski , \textbf{Dr Stephen Polyak}, Anna Keil , Michael Kantartjis , Shane Johnson , Dr Joshua Cosman, ``Wearable Sensor-Based Assessments for Remotely Screening Early-Stage Parkinson’s Disease",``Nature’s npj Parkinson’s Disease", 2023.
    \item[\textbf{--}] D. Anderson, M. Merickel, B. Severson, D. Amato, T. Kangarloo, J. Edgerton, R. Dorsey, J. Adams, S. Jezewski, A. Keil, S. Johnson, M. Kantartjis, \textbf{S. Polyak}, J. Severson, J. Cosman, ``WATCH-PD: Detecting Early-Stage PD using Feature Engineering and Machine Learning in Remote Sensor-Based Assessments",`` International Parkinson and Movement Disorder Society", Madrid, Spain, September, 2022.
    \item[\textbf{--}] Maris, Gunter, Michael Yudelson, and \textbf{Steve Polyak}, ``The ACT Master(y) Model for Learning, Measurement and Navigation",`` In Proceedings of HCI International", Copenhagen, Denmark, July, 2020.
    \item[\textbf{--}] Von Davier, Alina A., Benjamin Enver Deonovic, Michael Yudelson, \textbf{Steve Polyak}, and Ada Woo, ``Computational psychometrics approach to holistic learning and assessment systems." In Frontiers in Education, vol. 4, p. 69. Frontiers, 2019.
    \item[\textbf{--}]  Yudelson, Michael, Yigal Rosen, \textbf{Steve Polyak}, and Jimmy de la Torre. ``Leveraging Skill Hierarchy for Multi-Level Modeling with Elo Rating System." In Proceedings of the Sixth (2019) ACM Conference on Learning at Scale, pp. 1-4. 2019.
    \item[\textbf{--}] Von Davier, Alina A., Pak Chung Wong, \textbf{Steve Polyak}, and Michael Yudelson. ``The argument for a Data Cube for large-scale psychometric data." In Frontiers in Education, vol. 4, p. 71. Frontiers, 2019. Computational Psychometrics for the Measurement of Collab. Problem Solving Skills 2017
    \item[\textbf{--}] \textbf{Polyak, Stephen T.}, Alina A. von Davier, and Kurt Peterschmidt. ``Computational psychometrics for the measurement of collaborative problem solving skills." Frontiers in Psychology 8 (2017).
    \item[\textbf{--}] \textbf{Polyak, S.}, D. Edwards, A. Agrawal, J. Severson, K. Stoeffler, A. Best, and A. A. von Davier. ``Interpreting game log evidence of collaborative problem solving skills for middle school students using ML clustering techniques." In Poster presented at the thirtieth annual conference on neural information processing systems (NIPS), workshop on “Machine Learning in Education. 2016.
\end{itemize}
% }

\section{PATENTS AND AWARDS}
\item[]
    Corridor Leaders 250 - 2023 \vspace{-6pt}
    \begin{itemize}
        \resitem{\textbf{Recognition:} Corridor Business Journal 2023 - Corridor Leaders 250, Oct 2022. Selected as a featured area leader for the Iowa City - Corridor area for 2023.}       
    \end{itemize}

\item[]
    Corridor Leaders 250 - 2022 \vspace{-6pt}
    \begin{itemize}
        \resitem{\textbf{Recognition:} Corridor Business Journal 2022 - Corridor Leaders 250, Jan 2022. Selected as a featured area leader for the Iowa City - Corridor area for 2022.}       
    \end{itemize}

\item[]
    Systems and Methods for Interactive Dynamic Learning Diagnostics and Feedback 2019 \vspace{-6pt}
    \begin{itemize}
        \resitem{\textbf{Inventors:} Alina Von Davier, \textbf{Stephen Polyak}, Kurt Peterschmidt, Pravin Chopade, Michael Yudelson, Jimmy De La Torre, Pamela Paek. Publication date, 2019/5/2, US Patent office, Application number: 15802404}       
    \end{itemize}

\item[]
    Augmentation Award for Science and Engineering Research Training (AASERT) 
    \vspace{-6pt}
    \begin{itemize}
        \resitem{Recipient of the AASERT award from the Air Force Office of Scientific Research (AFOSR) USD \$83,000 tuition award.}       
    \end{itemize} 

\section{PRESENTATIONS}

\item[]
    CNS Summit, Boca Raton, Florida, USA 2021 
    \vspace{-6pt}
    \begin{itemize}
        \resitem{The Power of the Convergence of Digital Endpoints and Outcome Assessments, Clinical ink, Jonathan Andrus, Chief Strategy Officer \& Dr. Steve Polyak, VP, Engineering and Data}       
    \end{itemize} 

\item[]
    CNS Summit, Virtual, 2020 
    \vspace{-6pt}
    \begin{itemize}
        \resitem{Innovation Showcase: BrainBaseline, Digital Artefacts, Dr. Steve Polyak, VP, Engineering and Data}       
    \end{itemize} 

\item[]
    EdCrunch, Moscow, Russia 2019 
    \vspace{-6pt}
    \begin{itemize}
        \resitem{Talk: From Conceptualization to Implementation: Enabling a Scalable Voice-Based Learning Assistant. Also delivered at: ETCPS’19: Educational Technology and Computational Psychometrics Symposium Program in Iowa City, IA. (2019, October)} 
        \resitem{Panelist: Teacher vs Artificial Intelligence: When can Technology Teach Children Better?}      
    \end{itemize} 

\item[]
    IACAT, Minneapolis, MN 2019
    \vspace{-6pt}
    \begin{itemize}
        \resitem{Woo, Ada, Polyak, Steve T., Ou, Lu. Integrating Learning, Measurement, and Navigation: Introduction of a Recommendation and Diagnostics (RAD) API. IACAT 2019. Minneapolis}     
    \end{itemize} 
\item[]
    NCME, Toronto, Ontario, Canada 2019
    \vspace{-6pt}
    \begin{itemize}
        \resitem{Deonovic, Benjamin — Yudelson, Michael — Chopade, Pravin — Polyak, Steve. Toward dynamic adaptation and personalization in act academy–a free online learning platform. NCME 2019. Toronto, ON, Canada}     
    \end{itemize} 
\item[]
    International Test Commission. Montreal, Canada 2019
    \vspace{-6pt}
    \begin{itemize}
        \resitem{Polyak, S., Chopade, P., von Davier, A. A., and Woo, A. (2018). Blending learning and assessment: Introduction of a mobile platform that integrates test data and adaptive learning. 11th Conference of The International Test Commission. Montreal, Canada. Conference Presentation}     
    \end{itemize} 

\item[]
    NCME, New York, NY 2018
    \vspace{-6pt}
    \begin{itemize}
        \resitem{Deonovic, B., Polyak, S., and Yudelson, M. (2018). Modeling skill evidence from test preparation learning behaviors. 2018 NCME Annual Meeting. New York, NY. Conference Presentation}     
    \end{itemize} 

\item[]
    Michigan School Testing Conference, Ann Arbor, MI 2018
    \vspace{-6pt}
    \begin{itemize}
        \resitem{von Davier, A. A., Polyak, S. (2018, February) Analyzing game-based collaborative problem solving with computational psychometrics. Presented at Michigan School Testing Conference, Ann Arbor, MI}     
    \end{itemize} 
  
\item[]
    ACM Knowledge Discovery and Data Mining, Halifax, Nova Scotia, Canada 2017
    \vspace{-6pt}
    \begin{itemize}
        \resitem{ Polyak, S.T., von Davier, A.A., Peterschmidt, K. (2017, August) Computational Psychometrics for the Measurement of Collaborative Problem Solving Skills. In A. Lan (Chair) Advancing Education with Data. Workshop conducted at the 23rd ACM SIGKDD Conference on Knowledge Discovery and Data Mining, Halifax, Nova Scotia.}     
    \end{itemize} 
\item[]
    Harvard University, School of Education 2017
    \vspace{-6pt}
    \begin{itemize}
        \resitem{von Davier, A.A., Hao, J., Liu, L., Andrews, J., Halpin, P., Stoeffler, K., Polyak, S. (February, 2017). A Measurement Perspective on Collaborative Assessment Prototypes. Presentation at Harvard University, School of Education}     
    \end{itemize} 
\item[]
    Neural Information Processing Systems (NIPS) Barcelona, Spain 2016
    \vspace{-6pt}
    \begin{itemize}
        \resitem{Polyak, S., Edwards, D., Agrawal, A., Severson, J., Stoeffler, K., Best, A., Cosman, J., MacMillan, I., Dingler, C., Gambrell, J., von Davier, A.A. (December, 2016). Machine Learning Clustering Techniques: Interpreting Game Log Evidence of Collaborative Problem Solving Skills for Middle School Students. Presentation at the Neural Information Processing Systems conference (NIPS 2016), Machine Learning for Education Workshop. Barcelona, Spain.}     
    \end{itemize} 

\section{CERTIFICATIONS/TRAINING}

\item[]
    Generative AI with Large Language Models 2023 
    \vspace{-6pt}
    \begin{itemize}
        \resitem{Coursera: DeepLearning.AI \& Amazon Web Services}       
    \end{itemize} 
\item[]
    Artificial Intelligence (AI) in Python: A H2O Approach 2021 
    \vspace{-6pt}
    \begin{itemize}
        \resitem{Stack Skills, Issued Nov 2021}       
    \end{itemize} 

\item[]
    Biomedical (Biomed) Comprehensive 2021 
    \vspace{-6pt}
    \begin{itemize}
        \resitem{CITI Program, Expires Jul 2024}       
    \end{itemize} 
\item[]
    Researchers - Information Privacy and Security (IPS) 2021 
    \vspace{-6pt}
    \begin{itemize}
        \resitem{CITI Program, Expires Jul 2024}       
    \end{itemize} 
\item[]
    Databricks 2020 
    \vspace{-6pt}
    \begin{itemize}
        \resitem{Platform training for: SQL, ETL, MLFlow, Streaming Data}       
    \end{itemize} 
\item[]
    Data Visualization and D3.js 2018 
    \vspace{-6pt}
    \begin{itemize}
        \resitem{Udacity}       
    \end{itemize} 
\item[]
    Data Manipulation at Scale: Systems and Algorithms 2017 
    \vspace{-6pt}
    \begin{itemize}
        \resitem{Coursera, University of Washington}       
    \end{itemize} 
\item[]
    Protecting Human Research Participants 2017 
    \vspace{-6pt}
    \begin{itemize}
        \resitem{National Institute of Health}       
    \end{itemize} 
\item[]
    Machine Learning Specialization 2017 
    \vspace{-6pt}
    \begin{itemize}
        \resitem{Coursera, University of Washington, 4 courses}       
    \end{itemize} 
\item[]
    CodeSchool 2014 
    \vspace{-6pt}
    \begin{itemize}
        \resitem{Node.js, Rails, Javascript, HTML5, CSS, Sass, JQuery, CoffeeScript, Backbone.js}       
    \end{itemize} 

\section{Interests}
\begin{itemize}[leftmargin=10pt, noitemsep]
    \item[\textbf{--}] Cycling/Randonneuring/Bike Commuter
    \item[\textbf{--}] Regional Brevet Administrator (RBA) for the Iowa Randonneurs region
    \item[\textbf{--}] Artificial Intelligence
    \item[\textbf{--}] Arduino/Raspberry Pi
\end{itemize}

% \section{References}

% Available on Request

\end{document}